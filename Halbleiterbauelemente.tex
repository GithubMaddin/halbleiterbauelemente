% Fach
%----------------------------------------------------------------------------------------
%	PACKAGES AND OTHER DOCUMENT CONFIGURATIONS
%----------------------------------------------------------------------------------------

\RequirePackage[l2tabu, orthodox]{nag}  % Tells you when you use outdated packages
\documentclass[paper=a4, fontsize=11pt, titlepage=off]{scrartcl} % A4 paper and 11pt font size

\usepackage[utf8]{inputenc}  % Encoding
\usepackage[T1]{fontenc} % Use 8-bit encoding that has 256 glyphs
\usepackage{lmodern}
\usepackage[main=ngerman,english]{babel} % German/English language/hyphenation
\usepackage{amsmath,amsfonts,amsthm} % Math packages
\usepackage{mathtools}  % For prescript, mathclap etc. 
\usepackage{microtype}  % Makes more beautiful text
\usepackage{booktabs}  % Creates more beautiful Tables
\usepackage{graphicx}  % For graphics 
\usepackage{todonotes}  % Todo notes (\todo{Test})
\usepackage{ifthen}

\usepackage{color}
\usepackage{textcomp}
\usepackage{listings}  % Required for insertion of code
%\usepackage{lipsum} % Used for inserting dummy 'Lorem ipsum' text into the template

%\usepackage{sectsty} % Allows customizing section commands
%\allsectionsfont{\centering \normalfont\scshape} % Make all sections centered, the default font and small caps

%----------------------------------------------------------------------------------------
%	HEADER FOOTER
%----------------------------------------------------------------------------------------


%\usepackage{scrlayer-scrpage} % Custom headers and footers
%\usepackage{fancyhdr} % Custom headers and footers
%\pagestyle{fancyplain} % Makes all pages in the document conform to the custom headers and footers
%\fancyhead{} % No page header - if you want one, create it in the same way as the footers below
%\fancyfoot[L]{} % Empty left footer
%\fancyfoot[C]{} % Empty center footer
%\fancyfoot[R]{\thepage} % Page numbering for right footer
%\renewcommand{\headrulewidth}{0pt} % Remove header underlines
%\renewcommand{\footrulewidth}{0pt} % Remove footer underlines
%\setlength{\headheight}{13.6pt} % Customize the height of the header


\usepackage[footsepline]{scrlayer-scrpage}
%\renewcommand\sectionmarkformat{\thesection\autodot\quad}
\setkomafont{pageheadfoot}{\footnotesize\scshape}

%\automark{part}
%\automark*{section}
\clearpairofpagestyles% entfernen der voreingestellten Inhalte
\ifoot{\thesection}
\ofoot{\thepage}
\ihead{\customtitle}
%\ohead{}




%----------------------------------------------------------------------------------------
% SECTIONS
%----------------------------------------------------------------------------------------

\numberwithin{equation}{section} % Number equations within sections (i.e. 1.1, 1.2, 2.1, 2.2 instead of 1, 2, 3, 4)
\numberwithin{figure}{section} % Number figures within sections (i.e. 1.1, 1.2, 2.1, 2.2 instead of 1, 2, 3, 4)
\numberwithin{table}{section} % Number tables within sections (i.e. 1.1, 1.2, 2.1, 2.2 instead of 1, 2, 3, 4)

\setlength\parindent{0pt} % Removes all indentation from paragraphs - comment this line for an assignment with lots of text

\newcommand{\sectionruler}{\rule{0.45\linewidth}{1.5pt}}
\newcommand{\subsectionruler}{\rule{0.45\linewidth}{0.5pt}}
\let\stdsection\section
%\renewcommand\section{\ifthenelse{\equal{\arabic{section}}{0}}
%	{\stdsection}
%	{\newpage\stdsection\LARGE\arbic{\thesection}}

\renewcommand\section{\newpage\stdsection}
%\renewcommand*{\sectionformat}{%
%	\sectionruler\quad
%	\thesection
%	\quad\sectionruler
%	%\textls{}\enskip\chapternumbersize\thesection\\
%}
\setkomafont{section}{\rmfamily\Large\scshape}


%\renewcommand\thesection{\sectionruler\quad\arabic{UEBUNG}.\arabic{section}\quad\sectionruler}
%\renewcommand\thesubsection{\subsectionruler\quad\arabic{UEBUNG}.\arabic{section}.\arabic{subsection}\quad\subsectionruler}
%\renewcommand\thesection{\arabic{UEBUNG}.\arabic{section}}
%\renewcommand\thesubsection{\thesection.\arabic{subsection}}

%----------------------------------------------------------------------------------------
%	NEW COMMANDS
%----------------------------------------------------------------------------------------

%----------------------------------------------------------------------------------------
%	NEW ENVIRONMENTS
%----------------------------------------------------------------------------------------
\newenvironment{detpmatrix}{\left|\begin{pmatrix}}{\end{pmatrix}\right|}
\newenvironment{matrixarray}[1]{\left(\begin{array}{#1}\end{array}\right)}

%----------------------------------------------------------------------------------------
%	VARIABLES
%----------------------------------------------------------------------------------------

\newcounter{UEBUNG}
\newboolean{printNames}
\newboolean{printWithMatrikelNumbers}



%----------------------------------------------------------------------------------------
%	SETTINGS
%----------------------------------------------------------------------------------------

\newcommand{\customauthors}{}
\newcommand{\customMatrikel}{}
\newcommand{\customtitle}{Zusammenfassung Halbleiterbauelemente}  % TODO
\newcommand{\university}{Gottfried Wilhelm Leibnizuniversität Hannover}
\newcommand{\studiengang}{Technische Informatik}
\newcommand{\Uebungsstunde}{Martin Friedric, ...} % TODO
\newcommand{\Uebungsleiter}{}  % TODO


%----------------------------------------------------------------------------------------
%	PDF SETTINGS
%----------------------------------------------------------------------------------------


\usepackage[%
pdftitle={\customtitle},
pdfauthor={\customauthors},
pdfsubject={},   % TODO Subject
pdfcreator={LaTeX with KOMA-Script},
pdfkeywords={},
pdfpagemode=UseOutlines,
pdfdisplaydoctitle=true,
pdflang=de																 
]{hyperref}

%----------------------------------------------------------------------------------------
%	TITLE SECTION
%----------------------------------------------------------------------------------------



\newcommand{\horrule}[1]{\rule{\linewidth}{#1}} % Create horizontal rule command with 1 argument of height

\newcommand{\createtitle}{
\vspace*{\fill} % Comment for upper alignment
\thispagestyle{empty} % Comment to show footsepline and pagenumber
\setcounter{page}{0}
\begin{center}
	\normalfont \normalsize 
	\textsc{\university} % Your university, school and/or department name(s)
	\\ [15pt] % Your university, school and/or department name(s)
	\horrule{0.5pt} \\[0.4cm] % Thin top horizontal rule
	\huge \textsf{\customtitle} \\ % The assignment title
	\horrule{1.5pt} 
	\\[0.4cm] % Thick bottom horizontal rule
	%\normalsize
	\Large

	~\\[25pt]
	\large\textbf{\studiengang}\\
	\Large\textsf{\Uebungsstunde} \\
	\large\textsf{\Uebungsleiter} \\
	%\Large\today\\
	%\customauthors
	%\horrule{3pt} % Thick bottom horizontal rule
	%~\\[1cm]
\end{center}	
\vspace*{\fill}
}

\begin{document}
	%\maketitle
	\createtitle
		
	% Kapite 1
	\section{Kapitel 1 - Einführung}
	

\todo{In schönere Form bringen}

\subsection{Wichtige Daten}
\begin{itemize}
	\item Edison Effekt (1883)\\
	Gleichspannungsregler mit Vakuumdiode
	\item Triode (1920er-1930er)\\
	Glühfaden emitiert Elektronen, die beschleunigt durch ein Gitter treten und durch dieses gesteuert werden können.\\
	Hohe Stromverstärkung\\
	Hoher Stromverbrauch
	\item Feldeffekt Transistor nach Lilienfeld (1928)\\
	Konnte noch nicht herstellen, aber die Funktionsweise wurde beschrieben.\\
	Ge-Punktkontakt Transistor (1947)\\
	Current Transfer + resistor = Transistor
	\item MOSFET (1960)
	\item Erster Mikroprozessor (1971)
	\item Erster PC (1981)
\end{itemize}

\subsection{Moores Gesetz (1965)}
Alle 18 Monate verdoppelt sich die Anzahl der Transistoren pro Chip
Ableitung der heute gültigen Form im Rythmus von 2 Jahren.

\subsection{Skalieren}
Nach Moores Gesetzt halbiert sich die Fläche
=> Dimensionierung ändert sich um Faktor $1 \sqrt{2}$
Technologiegenerationen: 180 130 90 65 45 32 22 ...
Alte Technologien bleiben trotzdem erhalten
Betriebsspannung skaliert mit, zunächst Faktor 0.7 dann 0.85
Geschwindigkeit steigt exponentiell an
Preis pro Transistor fällt exponentiell ab
Preis einer Chipfabrik verdoppelt sich alle 3 Jahre

\subsection{Roadmap}
Von ITRS aufgestellt (Hersteller, Zulieferer, Forschung)
Gliedert in machbare, mit Forschung machbare und nicht nicht machbare Optionen
Man sieht eine "rote Mauer", die schon immer existiert hat und die sich alle 5-10 Jahre durch Fortschritt nach hinten verschiebt.

\subsection{Technologie}
Alles unter 100nm ist Nanotechnologie
MOS Logic und MOS Mikroprozessoren überliegen im Umsatzanteil.
Mehr als 90% Auf Silizium und mehr als 80% intigriert

\subsection{Gliederung}

Elektronische Bauelemente
Passive Bauelemente
Spulen
Widerstände
Induktivitäten
Aktive Bauelemente
Analoge Bauelemente
Digitiale Bauelemente
Kontakt- und Verbindungselemente
Leiterplatten
Schalter
Steckverbinder
Optoelektronische Bauelemente
Sender
Leuchtdiode
Halbleiter-Laser
Empfänger
Detektroren
Solarzellen

	
	% Kapitel 2
	\section{Kapitel 2 - Halbleitermaterialien}
	
\todo{Wichtige Begriffe erklären}
\subsection{Halbleiter}
	Die elektrische Leitfähigkeit von Halbleitern liegt zwischen der von Leitern und Nichtleitern. Mit steigender Temperatur nimmt ihre Leitfähigkeit zu. Primär existieren, im Gegensatz zu Metallen, keine freien Ladungsträger (diese entstehen erst z.B. durch thermische Anregung).
	Durch das Dotieren lässt sich die Leitfähigkeit jedoch gezielt beeinflussen.
	\begin{figure}[h]
		\centering
		\includegraphics[width=0.8\textwidth]{Kapitel/Kap02/halbleiter.PNG}
		\caption{Halbleiter}
		\label{02_HL}
	\end{figure}
	
\subsection{Eigenschaften von Silizium}
	\subsubsection{Stellung im Periodensystem}
		Silizium befindet sich in der 4. Hauptgruppe und der 3. Periode im Periodensystem und hat die Ordnungszahl 14.
		\begin{figure}[h]
			\centering
			\includegraphics[width=0.5\textwidth]{Kapitel/Kap02/elementhalbleiter.PNG}
			\caption{Silizium im Periodensystem}
			\label{02_elementHL}
		\end{figure}
		
		
	\subsubsection{Valenzelektronen}
		Silizium ist ein indirekter Halbleiter. Damit ein ELektron aus dem Valenzband in das Leitungsband übergehen kann, muss ihm neben einer Energie auch noch ein Impuls zugeführt werden. Diese Art der Übergänge sind energetisch wenig wahrscheinlich.
		Die benötigte Energie ist hier die Energiedifferenz von Leitungsband zu Valenzband :
		\begin{equation*}
		E_g = E_{LB} - E_{VB}
		\end{equation*}
		
		Der benötigte Impuls beträgt:
		\begin{equation*}
			\Delta p = 2\pi h(k_{LB}-k_{VB})
		\end{equation*}
		
		Silizium hat eine Bandlücke von ca 1.1eV.
		
		\newpage
		\begin{figure}[ht]
			\centering
			\includegraphics[width=0.5\textwidth]{Kapitel/Kap02/bandstruktur_SI.PNG}
			\caption{Bandstruktur von Silizium}
			\label{02_BS_SI}
		\end{figure}
		
	
	\subsubsection{Kristallstruktur}
		
		Kristallstruktur: Diamantstruktur
		Das Raumgitter besteht aus zwei kubisch-flächenzentrierten Gittern, die um 1/4 der Raumdiagonalen gegeneinander verschoben sind.
		Basis: identische Atome bei (0,0,0) und (1/4, 1/4, 1/4).
		Ein Si-Atom hat vier Außenelektronen, mit denen es Bindungen zu vier Nachbaratomen eingehen kann. Da (fast) alle Elektronen gebunden sind, ist die Leitfähigkeit (bei Raumtemperatur) sehr gering. (Das Valenzband ist (fast) vollständig besetzt, das Leitungsband hingegen (fast) vollständig leer.)
		
				
	\begin{figure}[h!]
		\centering
		\begin{minipage}[t]{0.35\linewidth}
			\centering
			\includegraphics[width=\linewidth]{Kapitel/Kap02/Diamantstruktur_SI.PNG}
			\caption{Diamantstruktur von Silizium}
			\label{02_diamStruktur}
		\end{minipage}% <- sonst wird hier ein Leerzeichen eingefügt
		\hfill
		\begin{minipage}[t]{0.35\linewidth}
			\centering
			\includegraphics[width=\linewidth]{Kapitel/Kap02/raeumlichesModell_SI.PNG}
			\caption{räumliches Modell von Silizium}
			\label{02_raeumlModell}
		\end{minipage}
	\end{figure}

		
	\subsubsection{Miller'sche Indizes und Kristallorientierungen}
	
		In der heutigen Mikroelektronik wird überwiegend Si(001) genutzt.
		Die Oberfläche von Si(001) hat eine 4-fache Symmetrie, damit gibt es zwei freie Bindungen pro Oberflächenatom.
		Die Oberfläche von Si(111) hat dagegen eine 3-fache Symmetrie und damit nur eine freie Bindung pro Oberflächenatom.
		
		\begin{figure}[h]
			\centering
			\includegraphics[width=0.22\textwidth]{Kapitel/Kap02/millerscheIndizes.PNG}
			\caption{Miller'sche Indizes}
			\label{02_millInd}
		\end{figure}
	
\subsection{Eigenleitung}
	T=0K: Alle möglichen Energiezustände im Valenzband sind von Elektronen besetzt. Das Leitungsband ist leer $\rightarrow$ es handelt sich um einen perfekten Isolator
	T>0K: Thermische Schwingungen können zur Bildung eines Elektron-Lochpaares führen. Die entstandenen freien Elektronen bedingen sich energetisch im Leitungsband. Die positiv geladenen Löcher verbleiben im Valenzband, sie sind dort ebenfalls beweglich.
	
	$Si \longrightarrow Si + e^- + h^+$ (e = electrons, h = holes)
	
	Im reinen Halbleiter finden Elektronenleitung \textbf{und} Löcherleitung statt. Die Anzahl von freien Elektronen und Löchern ist gleich: $\mathbf{n_i \cdot{p_i} = {n_i}^2}$
	
	\textbf{Generation:} Entstehung eines Elektron-Lochpaares
	\textbf{Rekombination:} Verschwinden eines Elektron-Lochpaares
	$\rightarrow$ hierbei fängt ein beliebiges Loch, unter Freigabe von Energie, ein beliebiges Elektron auf.
	
	Die Beweglichkeit von Elektronen ist immer sehr viel höher als die der Löcher.
	
\subsection{Dotierung}
	Die Eigenleitung von Halbleitern ist zu gering für eine technische Nutzung. Durch Dotierung (gezielte Verunreinigung mit Fremdatomen) wird die Leitfähigkeit erhöht. Die dotierten Materialien können überwiegend positive Ladungen (p-Typ) oder negative \textbf{freie} Ladungen (n-Typ) haben.
	Das \textbf{Massewirkungsgesetz} gilt aber auch für dotierte Halbleiter: $\mathbf{n \cdot{p} = {n_i}^2}$
	
	\subsubsection{Akzeptoren, Donatoren }
		\textbf{Donator:} ein Fremdatom der V-Hauptgruppe liefert ein zusätzliches Elektron
		Dotieren führt hier also zu einem Elektronenüberschuss $\rightarrow$ n-Leiter
		\begin{description}
			\item[$\bullet$] 5-wertige Stoffe sind z.B. Phosphor, Arsen, Antimon
			\item[$\bullet$] die abgegebenen Elektronen der Donatroren sind die (negativen) Ladungsträger
			\item[$\bullet$] es entstehen außerdem feste, positiv-geladene Ionen (tragen aber nicht zum Stromfluss bei)
			\item[$\bullet$] die Energieniveaus der Donatoren liegen dicht unterhalb des LB-Minimums in der Bandlücke ( $\rightarrow$ flache Störstelle)
		\end{description}
		
		\textbf{Akzeptor:} ein Fremdatom der III-Hauptgruppe kann ein Elektron aufnehmen
		Dotieren führt hier also zu einem Löcherüberschuss $\rightarrow$ p-Leiter
		\begin{description}
			\item[$\bullet$] 3-wertige Stoffe sind z.B. Bor, Gallium, ALuminium, Indium
			\item[$\bullet$] die entstehenden freien Löcher sind die (positiven) Ladungsträger
			\item[$\bullet$] es entstehen außerdem feste, negativ-geladene Ionen (tragen aber nicht zum Stromfluss bei)
			\item[$\bullet$] die Energieniveaus der Akzeptoren liegen dicht oberhalb des VB-Maximums in der Bandlücke ( $\rightarrow$ flache Störstelle)
		\end{description}
	
	\subsubsection{Konzentration, Lage im Bandgap usw.}
		
		
		\begin{figure}[h!]
			\centering
			\begin{minipage}[t]{0.45\linewidth}
				\centering
				\includegraphics[width=1.2\textwidth]{Kapitel/Kap02/tempAbhSiAs.PNG}
				\caption{Dotieratomanregung am Beispiel von SiAs}
				\label{02_tempAbh}
			\end{minipage}% <- sonst wird hier ein Leerzeichen eingefügt
			\hfill
			\begin{minipage}[t]{0.45\linewidth}
				\centering
				\includegraphics[width=1.2\textwidth]{Kapitel/Kap02/ladungstraegerkonzentration.PNG}
				\caption{T-Abhängigkeit der Ladungsträgerkonzentration (gilt auch für Akzeptoren)}
				\label{02_tempAbh2}
			\end{minipage}
		\end{figure}
		
		\begin{figure}[h!]
			\centering
			\begin{minipage}[t]{0.35\linewidth}
				\centering
				\includegraphics[width=\textwidth]{Kapitel/Kap02/tempAbh2.PNG}
				\caption{T-Abhängigkeit der Ladungsträgerkonzentration}
				\label{02_tempAbh3}
			\end{minipage}% <- sonst wird hier ein Leerzeichen eingefügt
			\hfill
			\begin{minipage}[t]{0.35\linewidth}
				\centering
				\includegraphics[width=\textwidth]{Kapitel/Kap02/tempAbh3.PNG}
				\caption{T-Abhängigkeit der Ladungsträgerkonzentration}
				\label{02_tempAbh4}
			\end{minipage}
		\end{figure}
		
		\newpage
		\textbf{Lage im Bandgap:}
		\begin{figure}[h!]
			\centering
			\includegraphics[width=0.9\textwidth]{Kapitel/Kap02/bandgapLage.PNG}
			\caption{Lage im Bandgap}
			\label{02_bandgap}
		\end{figure}
		
	\subsubsection{Dotantenaktivierung}
	DIe Anregung (Ionisation) wird als Aktivierung bezeichnet. Die Wahrscheinlichkeit für die Ionisation eines Donators ist von der Lage seines Energieniveaus $E_D$ abhängig. 
	Anzahl der Donatoren = neutrale + ionisierte ($N_D = {N_D}^0 + {N_D}^+$)
	\newpage
	
	\subsubsection{Dotiertechniken}
		\begin{figure}[h!]
			\centering
			\includegraphics[width=0.7\textwidth]{Kapitel/Kap02/dotiertechniken.PNG}
			\label{02_dotiertechniken}
		\end{figure}
		
\subsection{Gewinnung von Reinstsilizium}

	\subsubsection{Reduktion im Niederschachtofen}
		Sand (Siliziumdioxid - $SiO_2$) wird bei ca 1450°C geschmolzen. Dabei wird Kohlenstoff zugegeben. Der Kohlenstoff verbindet sich mit dem Sauerstoff zu Kohlenmonoxid, was so aus der Schmelze entweicht. Die Reaktionsprodukte (Silizium und gasförmiges Kohlenmonoxid) lassen sich leicht trennen.
		\begin{figure}[h!]
			\centering
			\includegraphics[width=0.4\textwidth]{Kapitel/Kap02/niederschachtofen.PNG}
			\caption{Niederschachtofen}
			\label{02_niederschachtofen}
		\end{figure}
		
	\subsubsection{Reinigung im Wirbelschichtreaktor}
		Das gewonnene Rohsilizium hat typischerweise immer noch Verunreinigungen von ca 2-4\% , die entfernt werden müssen. Hierfür wird das Rohsilizium zu Trichlorsilan umgewandelt und anschließend fraktioniert destilliert. Um das Silizium umzuwandeln wird es zu einer Korngröße von ca 0.1mm gemahlen und dann in einem Wirbelschichtreaktor mit Chlorwasserstoff durchwirbelt. Unter Wärmeentwicklung entsteht als Reaktionsprodukt hauptsächlich Trichlorsilan.
		Trichlorsilan hat einen Siedepunkt von 30°C und kann deshalb in großen, mehrstufigen Destillationsanlagen von den Verunreinigungen befreit werden. Dieses gereinigte Trichlorsilan dient als Ausgangsstoff für die Herstellung von polykristalllinem Reinsilizium.
		
	\subsubsection{Polyabscheidung (Siemensprozess)}
		Trichlosilan und Wasserstoff werden in einen Abscheidungsreaktor geleitet. Dieser besteht aus einer Quarzglocke, in dem sich eine u-förmige Brücke aus dünnen Reinstsiliziumstäben (Dünnstab) befindet. Hier wird polykristallines Silizium abgeschieden. Durch den Dünnstab wird ein elektrischer Strom geleitet, der für die nötige Erwärmung auf die Reaktionstemperatur sorgt. (Je dicker der Stab wird, desto höher wird der Strom geregelt.) Das polykristalline Silizium hat nun eine Reinheit von 99,9999999\% (9N).
\subsection{Einkristallines Silizium}
	Für die Chipherstellung wird einkristallines Silizium benötigt. Der Einkristall wird aus dem polykristallinen Silizium gezogen. Hierfür gibt es zwei Verfahren: Zonenziehen und Tiegelziehen im Schmelztiegel. Bei beiden Verfahren wird ein Impfkristall verwendet. Dies ist ein kleiner Einkristall mit einer genau definierten Ausrichtung des Kristallgitters. Die Kristallstruktur des erstarrenden Siliziums richtet sich genau an der des Impfkristalls aus $\Rightarrow$ Epitaxie
	
	\subsubsection{Zonenziehen und Tiegelziehen}
		\textcolor{red}{\textbf{Zonenziehen:}} (FZ-Verfahren)
			Silizium in Stabform wird mit einer Hochfrequenz-Induktionsspule direkt beheizt. Nach dem Anschmelzen wird der Schmelztropfen mit dem Impfkristall in Kontakt gebracht. Der sich drehende Stab wird durch die Spule abgesenkt und die Schmelzzone somit von unten nach oben gezogen.
			
			\textbf{Vorteile:}
				\begin{description}
					\item[$\bullet$] noch bestehenden Verunreinigungen werden in der Schmelze gelöst und so aus dem Stab herausgezogen
					\item[$\bullet$] Dotierungen, in Form von Prozessgasen, können gezielt vorgenommen und homogen in das Kristallgitter eingebaut werden
				\end{description}
			\textbf{Nachteile:}
			\begin{description}
				\item[$\bullet$] es lassen sich nur Stäbe von einem Durchmesser von 150mm sinnvoll ziehen
			\end{description}
				
			
		\textcolor{red}{\textbf{Tiegelziehen:}} (CZ-Verfahren)
		
		\begin{figure}[h!]
			\centering
			\includegraphics[width=\textwidth]{Kapitel/Kap02/tiegelziehen.PNG}
			\caption{Tiegelziehen}
			\label{02_tiegelziehen}
		\end{figure}
		
		\textbf{Vorteile:}
		\begin{description}
			\item[$\bullet$] ist vorallem für die Produktion von modernen großen Wafern geeignet (300mm und größer)
		\end{description}
		\textbf{Nachteile:}
		\begin{description}
			\item[$\bullet$] Dauer des Ziehprozesses: 1-3 Tage
			\item[$\bullet$] Inhomogenität bei der Vordotierung
			\item[$\bullet$] Verunreinigungen entstehen durch Kohlenstoff und Sauerstoff im Tiegel (die Verunreinigungen sind für viele Anwendungen aber gering genug)
		\end{description}
		
	\subsubsection{Zusammenfassung: Gewinnung von einkristallinem Silizium}
		\begin{description}
			\item[$\bullet$] \textbf{Gewinnung von Rohsilizium}
			\begin{description}
				\item[-] Reduktion im Niederschachtofen
				\item[-] Reinigung im Wirbelschichtreaktor $\rightarrow$ Umwandlung in Trichlorsilan
				\item[-] Polyabscheiung (Siemensprozess) $\rightarrow$ polykristallines Reinsilizium
			\end{description}
			\item[$\bullet$]Einkristallines Silizium
			\begin{description}
				\item[-] Zonenziehen oder Tiegelziehen
			\end{description}
			$\Rightarrow$ Einkristalline Si-Blöcke
		\end{description}
		
	\subsubsection{Wafermaße und -orientierungen}
		\begin{figure}[h!]
			\centering
			\includegraphics[width=0.4\textwidth]{Kapitel/Kap02/wafermasse.PNG}
			\caption{Flat oder Notch}
			\label{02_flat_notch}
		\end{figure}
		
		\begin{figure}[h!]
			\centering
			\includegraphics[width=0.6\textwidth]{Kapitel/Kap02/orientierungen.PNG}
			\caption{Waferherstellung: Flat}
			\label{02_flat}
		\end{figure}

	\todo{Fragen aus Own Clowd zuordnen}
	\todo{Gruppenübungs-Inhalte ergänze}

	
	% Kapitel 3
	\section{Kapitel 3 - Bandstruktur }
	\todo{Wichtige Begriffe erklären}
\subsection{Interferenz}
	\subsubsection{Verstärkung, Auslöschung, Weg- bzw. Phasenunterschied}
\subsection{Elektronen als Welle}
	\subsubsection{De Broglie Beziehung, Impuls, Wellenzahl usw.}
\subsection{Überlagerung von zwei Wellen}
	\subsubsection{Gleiche Wellen, entgegenlaufende Wellen, stehende Welle}
\subsection{Elektronen im Kastenpotential}
	\subsubsection{Einfacher Kasten, periodisches Kristallgitter}
\subsection{Bandstruktur von Halbleitern}
	\textbf{Bändermodell allgemein:}
	\begin{description}
		\item[Leitungsband:] Freie Elektronen (nicht mehr an ein bestimmtes Atom gebunden = Stromleitung möglich)
		\item[Bandlücke:] 'Verbotene Zone' $\rightarrow$ Energiebereich, ohne erlaubte Energieniveaus für Elektronen
		\item[Valenzband:] Energiebereich, in dem sich die äußeren Bindungselektronen des Atoms befinden (nicht für Stromleitung verfügbar) 
	\end{description}
	
	
	\subsubsection{Bandlücke, Brillouin-Zone
	Temperaturabhängigkeit}

		\textbf{Bandlücke bei Festkörpern:}
		\begin{description}
			\item[Metalle:] keine Bandlücke ($E_g$ = 0)
			\item[Isolatoren:] $E_g$ > 3eV
			\item[Halbleiter:] $E_g$ < 3eV
		\end{description}
		
		
\subsection{Direkte und indirekte Halbleiter}
	\begin{figure}[h!]
		\centering
		\includegraphics[width=\textwidth]{Kapitel/Kap03/uebergang_direkt_indirekt.png}
		\caption{direkter vs. indirekter Übergang}
		\label{02_uebergang}
	\end{figure}
	
	Für den Übergang von einem Elektron aus dem Valenzband ins Leitungsband ist bei einem direkten Halbleiter keine Impulsänderung notwendig; bei einem indirekten Halbleiter hingegen schon.
	
	\begin{figure}[h!]
		\centering
		\includegraphics[width=0.8\textwidth]{Kapitel/Kap03/direkt_indirekt.png}
		\caption{direkter vs. indirekter Halbleiter}
		\label{02_dir_ind_hl}
	\end{figure}
	
\subsection{Effektive Massen}
	Es kann an den Minima und Maxima im Bändermodell unertschiedliche $E(k)$ -Funktionen geben, dies nennt man entartet. 
	Je flacher das Leitungsband am Minimum ist, desto schwerer wird die effektive Masse der Elektronen. Es gibt also 'schwere' und 'leichte' Massen (z.B. hh : heavy holes und lh: light holes), es können demnach auch heavy holes und light holes zusammen an den Maxima und Minima auftreten.
	Alle Massen tragen gewichtet nach ihren Anteilen zum Ladungstransport bei. 
	$\rightarrow$ analoges gilt auch für Löcher
	
	\begin{figure}[h!]
		\centering
		\includegraphics[width=0.3\textwidth]{Kapitel/Kap03/effektiveMasse.png}
		\caption{Beispiel: GaAs}
		\label{02_effMasse}
	\end{figure}

	Die Geschwindigkeit der Ladungsträger ist proportional zur elektrischen Feldstärke:
	\begin{equation*}
		v_n = {-\mu}_n E \quad \textrm{ und } \quad v_p = {\mu}_p E
	\end{equation*}
	
	Die Proportionalitätskonstante wird als \textbf{Beweglichkeit} bezeichnet. Wobei die Beweglichkeit mit steigender Masse abnimmt: ${\mu} \sim \frac{1}{m^*}$ ($m^* = effektive Masse$)
	Die Beweglichkeit von Löchern und Elektronen ist unterschiedlich.
	
\subsection{Verspanntes Silizium}
	Verpsannt man Silizium, so hebt man die Entartung auf. Die Ladungsträger werden so energetisch günstiger für den Transport.
	\begin{figure}[h!]
		\centering
		\includegraphics[width=0.5\textwidth]{Kapitel/Kap03/verspanntesSi.png}
		\caption{Spannungseinfluss im Valenzband (Silizium)}
		\label{02_verspSi}
	\end{figure}
	
\todo{Fragen aus Own Clowd zuordnen}
\todo{Gruppenübungs-Inhalte ergänzen}
	
	% Kapitel 4
	\section{Kapitel 4 - Ladungsträger }
	\todo{Wichtige Begriffe erklären}
\subsection{Verteilungsfunktion}
	\subsubsection{Fermiverteilung}
	\subsubsection{Definition der Fermi-Energie}	
	\subsubsection{Lage des Fermi-Niveaus (intrinsisch vs. dotiert)}
	\subsubsection{Effektive Zustandsdichten}	
\subsection{Ladungsträger im Halbleiter}
	\subsubsection{Massenwirkungsgesetz}
	\subsubsection{Neutralitätsbedingung}
	\subsubsection{Intrinsische Ladungsträgerkonzentration}
	\subsubsection{Bezeichnung von dotierten Halbleitern}
	\subsubsection{Majoritäten und Minoritäten}
\subsection{Ladungsträgerbewegung}
	\subsubsection{Driftstrom, Sättigung usw.}
	\subsubsection{Diffusionsstrom}
	\subsubsection{Temperaturspannung}	
\subsection{Leitfähigkeit von Halbleitern}
	\subsubsection{p- und n-Typ, Temperaturabhängigkeit usw.}
	\subsubsection{Definitionen von Dotierniveaus}



\todo{Fragen aus Own Clowd zuordnen}
\todo{Gruppenübungs-Inhalte ergänzen}
	
	% Kapitel 5
	\section{Kapitel 5 - Ladungsträgerdynamik }
	\todo{Wichtige Begriffe erklären}
\subsection{Fermi-Niveau im Nichtgleichgewicht}
	\subsubsection{Gradienten und Ströme}
\subsection{Generation und Rekombination}
	\subsubsection{Überschuss-Ladungsträger, Lebensdauer}
	\subsubsection{Rekombinationsrate}
	\subsubsection{Verschiedene Rekombinationsmechanismen}
\subsection{Ortsabhängiges Energiebanddiagramm}
\subsection{Poisson-Gleichung}
\subsection{Kontinuitätsgleichung}
\subsection{Minoritätsladungsträger-Diffusion}
	\subsubsection{Lösungen für Spezialfälle}
\subsection{Zwei Halbleiter im Kontakt}
	\subsubsection{Offsets, Typ I, II, III}
	\subsubsection{Injektion, Akkumulation, Extraktion, Exklusion}

\todo{Fragen aus Own Clowd zuordnen}
\todo{Gruppenübungs-Inhalte ergänzen}
	
	% Kapitel 6
	\section{Kapitel 6 - Feldeffekttransistor }
	\subsection{Was bedeutet MOS-Technologie}
MOS steht für Metal Oxide Semiconductor.
Sie heißen auch unipolar: nur ein Ladungsträgertyp ist am Stromtransport beteiligt
	\subsubsection{NMOS/PMOS}
		NMOS bzw. PMOS sind nach der im Kanal entstehenden n- bzw. p-Leitung benannt.
	\subsubsection{CMOS}
		CMOS steht für Complementary MOS.
		Hierbei handelt es sich um eine kombinierte Verwendung von NMOS- und PMOS-Transistoren in einem Schaltkreis. Diese Anordnung ist stromsparend, hat eine geringe Verlustleistung und hoher Integrationsgrad.
\subsection{MOS-Kondensatoren}
	Die allgemeinere Bezeichung ist MIS-Kondensator. MIS - metal insulator semiconductor. Wobei der der Isolator  häufig aus Siliziumdioxid besteht
	\subsubsection{Grundaufbau}
	Der MOS-Kondensator besteht aus 
	\begin{itemize}
		\item Metallelektrode
		\item Isolator
		\item dotiertem Silizium		
	\end{itemize}	
	\begin{center}
		\includegraphics[width=0.2\linewidth]{Kapitel/Kap06/MOSKondensator}
	\end{center}
	
	\subsubsection{Flachbandfall}
		Background: im thermischen Gleichgewicht kommt es zu einem Ausgleich der unterschiedlichen Fermi-Niveaus (Halbleiter und Metall)
		\begin{itemize}
			\item Spannungsunterschied wird als Kontaktpotenzial bezeichnet
			\item \includegraphics[width=0.4\linewidth]{Kapitel/Kap06/Kontaktpotential} 			\includegraphics[width=0.4\linewidth]{Kapitel/Kap06/Kontaktpotential2}
			
			\item es bilden sich an der Grenzschicht zwischen Metall und Isolator positive und dem entgegengesetzt an der Grenzschicht zwischen Isolator und Halbleiter negative Ladungen aus.
			\item Es bildet sich eine Raumladungszone im Halbleiter			 		
		\end{itemize}
		Wichtig: Gleicht man das Kontaktpotenzial und den Spannungsabfall über das Oxid durch eine bestimmte schwache 	Vorspannung aus, so verschwindet die Raumladungszone man spricht vom Flachbandfall und der angelegten Flachbandspannung $U_{FB}$.
		
	\newpage
		
	\subsubsection{Anreicherung/Akkumulation, Verarmung, Inversion}
		Arbeitszustände (in diesen Fällen für für p-Silizium Substrate):
		\newline
		!!! Diagramme dienen nur der Veranschaulichung, Inversionsdiagramme sind wichtig !!!!
		\newline
		\textbf{Akkumulation:}
		\newline
		Beim Anlegen einer negative Spannung ($U_{MS} < 0 V $) gegenüber dem Substrat wandern die positiven Ladungsträger im Substrat zur 		Grenzschicht, sammeln sich dort und bilden eine Anreicherungszone
		\newline
		\includegraphics[width=0.25\linewidth]{Kapitel/Kap06/Akkumulation1}		
		\includegraphics[width=0.60\linewidth]{Kapitel/Kap06/Akkumulation}
		\newline
		\newline
		
		\textbf{Verarmung:}
		\newline
		Bei Anlegen des Pluspols am Metall und
		des negativen Pols am Substrat ($U_{MS} > 0V$) wandern negative Ladungs-träger
		(Minoritäten) im Substrat an die Grenzschicht und rekombinieren mit den dort befindlichen freien positiven Ladungsträgern.
		\newline 
		In der Nähe der Grenzschicht entsteht durch die Rekombinationen eine 		Raumladungszone, die an freien 		Ladungsträgern verarmt ist. Diese Zone wird als Verarmungszone bezeichnet.
		\newline
		\includegraphics[width=0.25\linewidth]{Kapitel/Kap06/Verarmungszone1}
		\includegraphics[width=0.60\linewidth]{Kapitel/Kap06/Verarmungszone2}
		
		\textbf{Inverstion:}
		\newline
		Überschreitet die angelegte Spannung eine Schwellspannung ($U_{MS} > U_{th}$ ,Threshold) bildet sich im ursprünglich p-dotierten Substrat ein n-dotiertes Gebiet. Es stehen keine freien Löcher mehr an der Grenzschicht zur Rekombination zur Verfügung.
		\newline
		Die so entstandene Zone, die die frei beweglichen negativen Ladungsträger enthält, wird als Inversionszone 	bezeichnet.
		\newline
		\includegraphics[width=0.25\linewidth]{Kapitel/Kap06/Inverstion1}		
		\includegraphics[width=0.60\linewidth]{Kapitel/Kap06/Inversion2}
		
		
\subsection{MOS-Transitoren}
MOS Transistoren sind Feldeffekttransistoren =
MOSFET
	\subsubsection{Grundaufbau}	
	Ein MOSFET ist ein aktives Bauelement mit mindestens drei Anschlüssen:
	\begin{itemize}
		\item S (source, dt. Quelle)
		\item D (drain, dt. Abfluss)
		\item G (gate, dt. Steuerelektrode)
		\item (B (bulk, dt. Substrat), Bei einigen Bauformen wird ein zusätzlicher Anschluss B nach außen geführt. Meistens ist das 		Substrat jedoch intern mit S verbunden )
	\end{itemize}
	Betrieb: Majoritätsladungsträger fließen von S nach D:
	\begin{itemize}
		\item unipolares Bauelement
		\item laterales Bauelement
	\end{itemize}
	Basierend auf dem Feldeffekt wird der Stromfluss wird durch ein an G anliegendes	elektrisches Feld gesteuert.
	\newline
	Die Geschwindigkeit ist vom Ladungsträgertransport von Source zum Drain abhängig. Dabei liegen die heutigen Kanallängen bei < 32 nm.
		%MOS Transistor
		\begin{center}
			\includegraphics[width=0.7\linewidth]{Kapitel/Kap06/MOS_Transistor.png}
		\end{center}
	\subsubsection{neuere Entwicklungen}
		\textbf{Problem: }
		\newline
		Bei der Skalierung zu neuen Technologiegenerationen wird auch die Gateoxiddicke ebenfalls reduziert. 
		Die Tunnel/Leak-Ströme steigen exponentiell mit abnehmender Dicke.
		Irgendwann wäre man bei 3 Lagen von SiO2-Tetraedern. Die ist technisch nicht mehr homogen realisierbar (min. Schwankung 33 \%), da auch nicht mehr messbar. 
		\newline
		\textbf{Lösung: }
		\newline
		Daher muss trotz Skalierung das Dielektrikum dicker sein. Um die Kapazität bei größerer Dicke beizubehalten werden sogenannte High-K Dielektrika eingesetzt. 
		
		$C = \frac{\epsilon_r  \epsilon_0 A}{d}$ => Wird die Dicke d erhöht muss ein Material mit höherer Dielektrizitätskonstante $\epsilon_r$ (oder amerikanisch K) gefunden werden, um dieses auszugleichen.
	\subsubsection{Die vier Grundtypen}
	\subsubsection{Funktionsweise eines n-Kanal-Anreicherung-Transistoren}
	\subsubsection{Ausgangs- und Transferkennlinien}
\subsection{Anwendungen von MOS-Transistoren}
	
	\subsubsection{Vor- und Nachteile}
	\subsubsection{Funktionsprinzip eines CMOS-Inverter}


\todo{Fragen aus Own Clowd zuordnen}
\todo{Gruppenübungs-Inhalte ergänzen}
	
	% Kapitel 7
	\section{Kapitel 7 - Speicherbauelement }
	\todo{Wichtige Begriffe erklären}
\subsection{Flüchtige Speicher}
	Bei flüchtigem Speicher handelt es sich um Speicher, die ihre Information ohne Spannungsversorgung verlieren.
	Das Random-Access-Memory(RAM) steht für den wahlfreien Zugriff, also dass jede Speicherzelle über die Speicheradresse direkt angesprochen werden kann. Man unterscheidet zwischen zwei Arten:
	\subsubsection{SRAM}
		Beim SRAM wird die Ladung in einem Flip-Flog (2 CMOS Inverter) gespeichert. Diese werden mit 6 Transistorzellen realisiert. 
		\newline
		% Bild SRAM Zelle
		\begin{center}
			\includegraphics[width=0.3\linewidth]{Kapitel/Kap07/SRAMZelle.png}
		\end{center}		
		Vorteil: sehr schnell, behält Information auch ohne Taktspannung, kein Refresh nötig
		\newline
		Nachteil: geringere Speicherdichte
		\newline
		Mit Hilfe einer Pufferbatterie kann der SRAM zu einem nichtflüchtigen Speicher gewandelt werden.
	\subsubsection{DRAM}
		Die Informationen werden in Form des Ladezustandes eines  MOS-Kondensator gespeichert. Die Realisierung benötigt daher nur 1 Transistorzelle und einen Kondensator.
		\begin{center}
			\includegraphics[width=0.2\linewidth]{Kapitel/Kap07/DRAMZelle.png}
		\end{center}
		Der Kondensator entlädt sich bei den kleinen möglichen Kapazitäten durch die auftretenden Leckströme schnell. Ein Refresh mit Taktspannung etwa alle 32 ms ist notwendig. Das Lesen ist destruktiv. Auch hier ist ein Refresh notwendig.
		\newline
		Vorteil: wenig Platzbedarf, gute Skalierbarkeit, kostengünstig
		\newline
		Nachteil: Refresh nötig
		\newline
		
		\textbf{Technische Realisierung}
		\newline
		Die technische Realisierung des Kondensatoren wurde in der Vergangenheit auf der Oberfläche des Halbleiters realisiert und war damit sehr platzintensiv. 
		\begin{center}
			\includegraphics[width=0.3\linewidth]{Kapitel/Kap07/Condensatorimplmentierung_Alt}
		\end{center}
		
		Heutzutage findet die Implementierung entweder in Form eines Grabens (Trench) statt, wobei der Kondensator durch Ätzen eines 5-10 $\mu m$ tiefen Loches im Substrat erzeugt wird. 
		\newline
		Alternativ gibt es die Stacked Form, wobei der Kondensator über dem Transistor aufgebaut (gestapelt) wird.
		
		\begin{center}
			\includegraphics[width=0.4\linewidth]{Kapitel/Kap07/TrenchAndStack}
		\end{center}
		

\subsection{Nichtflüchtige Speicher}
	Nichtflüchtiger Speicher bewahren gespeicherte Information auch ohne Spannungsversorgung.
	Auch hier gibt es verschiedene Formen:
	\subsubsection{ROM}
		\begin{itemize}
			\item Datenspeicher, der nur lesbar ist, im normalen Betrieb aber nicht beschrieben werden kann.
			\item Information meist bei der Chip-Produktion vom Hersteller eingeschrieben
		\end{itemize}
	\subsubsection{PROM}
		\begin{itemize}
			\item einmalige Programmierung durch Anwender
			\item Schreiben destruktiv und nicht reversibel (z.B. Durchbrennen spezieller Leitbahnen)
			\item Programmierung durch Ausbrennwiderständen oder kurzgeschlossene Sperrschicht
		\end{itemize}
	\subsubsection{EPROM}
		\begin{itemize}
			\item beschränktes Löschen und Programmieren
			\item langsames Schreiben erfolgt elektrisch
			\item  Löschen (meist komplett) erfolgt z.B. mit UV-Licht
		\end{itemize}
	\subsubsection{EEPROM}
		\begin{itemize}
			\item einzelne Speicherzellen können elektrisch gelöscht und wieder programmiert werden		
			\item zusätzliche Auswahltransistoren für die Zellen
			\item Die Be- und Endladung erfolgt i.d.R. über ein Floating Gate (bewirkt Verschiebung der Schwellspannung am Steuergate)
			\begin{center}
				\includegraphics[width=0.5\linewidth]{Kapitel/Kap07/FloatingGate}
			\end{center}
			\item Die Anzahl der möglichen Schreibvorgänge ist auf Grund der Abnutzung (Oxiddegradationseffekten) beim Schreiben allerdings begrenzt
		\end{itemize}
	\subsubsection{Flash-EEPROM}
		\begin{itemize}
			\item portable und miniaturisierte Weiterentwicklung des EEPROMs
			\item Um die Ladungen auf das Floating Gate zu bringen wird beim Flash ein quantenphysikalischen
			Tunneleffekts (Fowler-Nordheim-Tunneln) ausgenutz, der es den Elektronen erlaubt, den Tunnelisolator zu passieren.
			\item Dies erfordert große Unterschiede im elektrischen Potential über den Isolator
		\end{itemize}
	\subsubsection{Solid State Devices}
		\begin{itemize}
			\item SSDs speichern Daten in Flash-Bausteinen
			\item Sie sind teurer aber robuster als Festplatten
			\item SSDs besitzen auf Grund möglicher Defekte des Flash Speichers in der Regel mehr Speicherzellen, um defekte Zellen zu ersetzen 
		\end{itemize}
	
	\subsubsection{Neuere Lösungen}
	\begin{itemize}
		\item Multibitspeicher (MLC)
			\begin{itemize}
				\item Es werden mehrere Bits pro Speichertransistor gespeichert
				\item Man nutzt hierzu verschiedene Ladungszustände des Transistors bzw. dessen elektrische Leitfähigkeit
			\end{itemize}
		\item Nanoclusterspeicher
		\begin{itemize}
			\item Aktuelle Floatinggates können bei einem Defekt des Tunnelisolators ihren Zustand nicht halten 
			\item Lösung: Nanocluster (Kugeln) ohne Querleitfähigkeit zwischen den einzelnen Clustern
		
			\includegraphics[width=0.3\linewidth]{Kapitel/Kap07/Nanocluster}
			
		\end{itemize}
		\item MRAM und 
			\begin{itemize}
				\item Bisherige Magnetisierung von Magnetspeichern aus der Ebene drehen
				\item ermöglicht Miniaturisieren und Skalierbarkeit
				\item Informationen werden nicht mit elektrischen sondern mit magnetischen Ladungselementen gespeichert
			
				\includegraphics[width=0.3\linewidth]{Kapitel/Kap07/MRAM.png}
				
			\end{itemize}
		\item FeRAM
			\begin{itemize}
				\item Der Aufbau entspricht dem einer DRAM-Zelle. Anstelle eines konventionellen Kondensators wird ein Kondensator mit ferroelektrischem Dielektrikum eingesetzt
				
				\includegraphics[width=0.2\linewidth]{Kapitel/Kap07/FeRam.png}
				
			\end{itemize}
	\end{itemize}
	

	
	





\todo{Fragen aus Own Clowd zuordnen}
\todo{Gruppenübungs-Inhalte ergänzen}
	
	% Kapitel 8
	\section{Kapitel 8 - pn-Übergang }
	\todo{Wichtige Begriffe erklären}

\subsection{pn-Übergang}
	\subsubsection{Grundverhalten}
	 	Ein PN-(Homo-)Übergang liegt vor, wenn zwei Halbleiter aus dem selben Material in Kontakt sind, deren Dotierung unterschiedlich ist. 
	 	\newline
	 	Im thermischen Gleichgewicht ohne externes Feld  	bleibt das Ferminiveau konstant (Pinning)
	 	\newline
	 	\begin{center}
	 		\includegraphics[width=0.6\linewidth]{Kapitel/Kap08/PN-uebergang_1.png}
		 	\includegraphics[width=0.35\linewidth]{Kapitel/Kap08/PN-uebergang_2.png}
		\end{center}
 	\subsubsection{metallurgische Grenze}
 		Der Punkt M, an dem die Dotierung sich von p zu n  	ändert, wird metallurgische Grenze genannt.
 		\begin{center}
 			\includegraphics[width=0.4\linewidth]{Kapitel/Kap08/metallurgischeGrenze}
 		\end{center}
 		
 	\subsubsection{Diffusion und Drift}
 		\begin{enumerate}
 			\item Aufgrund des hohen Ladungsträgerkonzentrationsunterschiedes
 			kommt es zur Diffusion:
 			\begin{itemize}
 				\item von freien Löchern aus dem p-Gebiet ins n-Gebiet.
 				\item und von freien Elektronen aus dem n-Gebiet ins p-Gebiet
 			\end{itemize}
 			\item In der Nähe der Grenze M rekombinieren dann die diffundierten Ladungsträger mit den dortigen Majoritätsträgern.
 			\item Es kommt zu einer Verarmung der freien Ladungsträgern
 			\item Durch die verbleibenden festen Akzeptor-Ionen (p-Gebiet) und Donator-Ionen (n-Gebiet) entsteht ein lokales E-Feld entgegen der Diffusionsrichtung der Ladungsträger. Dies führt zu Driftströmen der freien Ladungsträgern.
 			\item Es stellt sich ein Gleichgewicht ein, so dass sich die Ladungsträgerströme resultierend aus Drift und Diffusion genau kompensieren.
 		\end{enumerate}
 		%drift und diffusion
		 \begin{center}
		 	\includegraphics[width=0.4\linewidth]{Kapitel/Kap08/driftUndDiffusion}
		 \end{center}

	\subsubsection{Raumladungszone / Verarmungszone}
		Die Raumladungszone entsteht durch die Rekombination der freien Ladungsträger an der metallurigischen Grenze. Hierbei kommt es zur Verarmung der freien Ladungsträger, weshalb diese auch Veramungszone genannt wird. 
	\subsubsection{Eingebautes Feld}
	
		\begin{center}
			Zwischen den Festen Donator und Akzeptor-Ionen entsteht im Bereich der Raumladungszone ein elektrisches Feld.
			\includegraphics[width=0.4\linewidth]{Kapitel/Kap08/eingebautesFeld.png}
		\end{center}
		Den maximalen Wert $E_0$ des negativen elektrischen Feldes erreicht man an der Stelle der metallurgischen Grenze
		\newline
		Das elektrische Feld im thermischen Gleichgewicht wird eingebautes Feld genannt.
		
	\subsubsection{Diffusionsspannung}
		Die Spannung, die im thermischen Gleichgewicht über den pn-Übergang abfällt ($V_0$), wird auch 		eingebaute Spannung oder Diffusionsspannung ($U_D$) genannt.
		
		$V_0 = - \frac{E_0*W}{2}$
		
	\subsubsection{Breite der Raumladungszone}
		Die Breite der Raumladungszone hängt von den Donator- und Akzeptorkonzentrationen ab. Es gilt:			
		\begin{center}
			\includegraphics[width=0.5\linewidth]{Kapitel/Kap08/breiteRaumladungszoneNeutralisaetsbedingung.png}	
		\end{center}
	
		Daraus ergibt sich:	Für $N_A > N_D$ gilt $W_n > W_p$
		\newline
		Die Ausdehnung erfolgt immer stärker in das niedriger
		dotierte Gebiet.
		\newline
		Die Breite der Raumladungszone sinkt mit zunehmender Dotierung des n- bzw. p-Gebietes.
		\newline
		\textbf{Größenordnung Raumladungszone:}
		
		Annahme: Silizium mit p-Dotierung = n-Dotierung = $10^{18} cm^{-3}$
		
		Raumladungszone $W ca. 1 \mu m$
	
	\subsubsection{pn-Übergang mit externer Spannung}
	Externe Spannung V kann W vergrößern oder verkleinern:
	\begin{itemize}
		\item Für $V > V_0$ verschwindet die Raumladungszone -> Flussspannung 
		\item Für eine negative Spannung vergrößert sich die Raumladungszone -> Sperrspannung
	\end{itemize}
	\begin{center}
		\includegraphics[width=0.6\linewidth]{Kapitel/Kap08/pnUebergangExterneSpannung}
	\end{center}
	\begin{center}
		Gleichgewicht, Flussrichtung, Sperrrichtung
	\end{center}
	
\subsection{pn-Diode}
	Bei einer pn-Diode treffen ein p und ein n-Gebiet aufeinander. Durch Ablegung einer Spannung kann die Diode den Strom leiten, da die Raumladungszone gegen Null geht.
	\begin{center}
		\includegraphics[width=0.3\linewidth]{Kapitel/Kap08/Diode}
	\end{center}

	\subsubsection{Kennlinien für verschiedene Halbleitermaterialien}
		Größere Bandlücke verschiebt das Einsetzen des Stromes zu höheren Spannungen.
		\begin{center}
			\centering
			\includegraphics[width=0.7\linewidth]{Kapitel/Kap08/DiodeVerschiedeneHLMat.png}	
		\end{center}
	
	
	\subsubsection{Flusspolung, Sperrrichtung}
		\textbf{Flusspolung: }
		\begin{center}
			\includegraphics[width=0.5\linewidth]{Kapitel/Kap08/PNUebergangFlusspolung}
		\end{center}
		In Flussrichtung wird die Raumladungszone verringert und es kommt zu einem Stromfluss
		\newline
		\textbf{Sperrichtung:}
		\begin{center}
			\includegraphics[width=0.5\linewidth]{Kapitel/Kap08/sperrichtung}
		\end{center}
		In Sperrichtung ist kein großer Strom aufrecht zu
		erhalten. Es entsteht trotzdem ein kleiner Strom (Sperrstrom), da Minoritätsträger
		in die Raumladungszone diffundieren und dann
		durch das elektrische Feld ins gegenüberliegende
		Gebiet transportiert werden.
		Der Sperrstrom kann noch weitere Bestandteile haben, z. B. Oberflächenleckströme
		
	\subsubsection{Ideale Diodengleichung}
		\begin{center}
			Ideale Dioden-Gleichung oder Shockley-Gleichung.
			Wichtig hierbei ist, dass der Strom proportional exponentiell von der Bandlücke des Materials abhängt.
			\includegraphics[width=0.3\linewidth]{Kapitel/Kap08/IdealeDiodengleichung.png}
			\includegraphics[width=0.2\linewidth]{Kapitel/Kap08/IdealeDiodengleichungProportionalitaet.png}	
		\end{center}
	
	\subsubsection{Ideal vs. Real}
		\begin{center}
			\includegraphics[width=0.25\linewidth]{Kapitel/Kap08/IdealvsReal.png}
			\includegraphics[width=0.35\linewidth]{Kapitel/Kap08/RealeDiode}
			
		\end{center}

	\subsubsection{Idealitätsfaktor}
		In realen pn-Übergängen tritt eine gewisse Rekombination in der Raumladungszone auf, was zu einem zusätzlichen äußeren Strom führt. 
		Gesamtstrom in einer pn-Diode ergibt sich daher aus der Summe des idealen Diodenstromes mit dem
		Rekombinationsstrom in der RLZ:
		\newline
		Bei der Beschreibung von realen Diodencharakteristiken in bestimmten Spannungsbereichen wird dabei folgende Gleichung verwendet (Gleichung irrelevant)
		\begin{center}
			\includegraphics[width=0.3\linewidth]{Kapitel/Kap08/genaherteGleichung}
		\end{center}
		Der darin enthaltene \textbf{ Faktor $\eta$ wird als Idealitätsfaktor bezeichnet und liegt immer zwischen 1 und 2}
		\begin{center}
			\includegraphics[width=0.5\linewidth]{Kapitel/Kap08/IdealitaetsfaktorBeispiel}
		\end{center}
		
	\subsubsection{Temperaturabhängigkeiten}
		Bei wachsender Temperatur nimmt der ideale
		Stromanteil gegenüber dem nichtidealen zu. Die Einsatzspannung der Diode sinkt.
		\begin{center}
			\includegraphics[width=0.25\linewidth]{Kapitel/Kap08/temperaturabhaengigkeit1}
		\end{center}
		Mit wachsender Temperatur nimmt auch der Sperrstrom zu
		\begin{center}
			\includegraphics[width=0.25\linewidth]{Kapitel/Kap08/temperaturabhaengigkeit2}
		\end{center}
		
	
	\subsubsection{Durchbruch}
	Die pn-Diode kann nicht mit beliebig großer Spannung in Sperrichtung betrieben werden.
	Eine zu hohe Spannung in Sperrichtung führt dazu, dass die Diode "durchbricht". Der Strom steigt ab einer kritischen Sperrspannung (Durchbruchsspanung) sehr stark an. 
	\subsubsection{Lawinendurchbruch}
	Durch ausreichend hohe Sperrspannungen ist es 	möglich, die Raumladungszone und damit das 	elektrische Feld am pn- Übergang soweit zu vergrößern, 	dass die beschleunigten Ladungsträger ausreichend hohe Energien erreichen, um ihrerseits durch Stöße mit Kristallatomen weitere Elektron-Loch-Paare zu erzeugen (Stoßionisation).
	Die so generierten Elektronen und Löcher können bei ausreichender Beschleunigung im elektrischen Feld ihrerseits wieder Stoßionisationprozesse auszulösen.
	Somit kommt es zu einer Lawinenartigen Bildung von Ladungsträgern, der Sogenannte Lawinendurchbruch.
	\subsubsection{Zener-Effekt/-Diode}
	Der Zener-Effekt tritt bei hoch
	dotierten pn-Übergängen auf, in denen schmale Raumladungszonen und hohe elektrische Feldstärken auftreten.
	In diesem Fall kann es unter Sperrspannung zum Tunneln von Elektronen des Valenzbandes der p-Seite ins Leitungsband der n-Seite kommen, wobei auch Elektron-Lochpaare in der RLZ geschaffen werden. Dies führt zu einem Starken Anstieg des Stroms. Die kritische Spannung liegt dabei niedriger als die Durchbruchsspannung. 
	\newline
	Die auf diesem Effekt basierende Z-Dioden erlaubt es, dass sie in zahlreichen Schaltungen zur Stabilisierung und
	Begrenzung von elektrischen Spannungen eingesetzt werden.
	\begin{center}
		\includegraphics[width=0.3\linewidth]{Kapitel/Kap08/Zenerdiode}
	\end{center}
	
	\subsubsection{Varaktordioden}
		In Sperrrichtung wirkt die Sperrschicht bzw. Raumladungszone am pn-Übergang wie eine Kapazität.
		Ändert sich die Spannung an der Diode ändert sich auch die Kapazität der Sperrschicht.
		Die Varaktordiode wird daher als spannungsabhängige Kapazität (Varaktor) genutzt.
		Hierbei ist die Sperrschicht-Kapazität
		besonders groß. Dadurch sind große Kapazitätsänderungen möglich.
		\begin{center}
			\includegraphics[width=0.3\linewidth]{Kapitel/Kap08/Varaktordiode}
		\end{center}
		
	
	\subsubsection{Diodentypen und ihre Anwendungsfelder}
		Vermutlich eher unwichtig:
		\begin{itemize}
			\item Optik -> Laserdiode, Fotodiode, LED
			\item Kapazitive Dioden
			\item Gesteuerte Gleichrichter und verwandte Bauelemente -> Vierschichtdioden, Thyristor
			\item diverse weitere
		\end{itemize}


\todo{Fragen aus Own Clowd zuordnen}
\todo{Gruppenübungs-Inhalte ergänzen}
	
	% Kapitel 9
	\section{Kapitel 9 - Solarzelle }
	\todo{In schönere Form bringen}
\todo{Zusammenfassen}

\subsection{Auslaufen der Energie Reserven}
Kohle ca. 100 Jahre
Erdgas und Erdöl ca. 50 Jahre
Uran ca. 70 Jahre

\subsection{Anteile an Erneuerbaren Energie}
Vor allem Skandinavien
Deutschland 38,5\% am gesamten Strommix

\subsection{Massendefekt}
Masse $\leftrightarrow$ Energie: $E = m\cdot c^2$
Massendefekt = Bindungsenergie
Diese Energie wird bei Kernumwandlungen freigesetzt

\subsection{Kernfusion}
$Deuterium H^3 + Tritium H^2 \rightarrow Heliumkern He^2 + n$

\subsection{Geschichte}
Erste Solarzelle (1954)
4-6\% Wirkungsgrad

\subsection{Vorteile Solarenergie}
Unbegrenzt vorhanden
Emissionslos
Reduzierung von energiepolitischen Abhängigkeiten

\subsection{Nachteile Solarenergie}
Nicht konstant (Wetter, Jahreszeiten, Tageszeiten...)
Herstellung nicht emissionsfrei
Hohe Kosten

\subsection{Brandenburg ist toll}

\subsection{Aufbau}
95% aler Solarzellen aus Silizium
np-Halbleiter Aufbau
n Schicht ist sehr dünn, damit das Licht an den pn-Übergang gelangen kann

\subsection{Photoelektrischer Effekt}
Äußerer (nicht für Solarzellen relevant)
Aufgeladene Oberfläche gibt Elektronen bei Bestrahlung ab
Innerer (relevant für Solarzellen)
besser Leitfähigkeit bei Beleuchtung, da Elektronen auf höheres Valenzband gehoben werden

\subsection{Ablauf}
Licht sorgt dafür, dass Elektronen in die n Schicht und Löcher in die p Schicht wandern. Können an Kontakten abgegriffen werden.
Elektronen-Loch-Paare müssen vor der Rekombination getrennt werden
Spannung immer ähnlich (0.5 V bei Silizium), Strom steigt mit Beleuchtungsstärke an
Leistung ist temperaturabhängig

$Stromabgabe = Generation - Rekombination = J = J_SC - J_0 \bigg(e^{\frac{qV}{kT}}-1\bigg)$
Rekombination = Dunkel-Diodenstrom
Keine Annahme über pn- Übergang für die Formale notwendig

Der pn-Übergang ist für die Kontaktierung. Die Fermi.Level der Metallkontakte richtet sich nach dem der angrenzenden Majoritäten
Somit könne die Löcher durch p und Elektronen durch n Halbleiter kontaktiert werden. Fließen nach außen ab.

Kennlinie wie bei einer Diode, aber verschiebt sich bei Sonneneinstrahlung

Maximum Power Point bei P=J*V = max
Füllfaktor ist Verhältnis zwischen Leistung und MPP

\subsection{Wirkungsgrad}
MPP / Lichtintensität
Photonen nicht energetisch genug ($hv < E_g$)
Photonen haben Überschussenergie die in Wärme umgesetzt wird ($hv > E_g$)
Abschattung durch Kontakte
Widerstandsverluste
Es kann nicht das gesamte Spektrum genutzt werden
Theoretisch auf 32,2\% bei Licht von 1,12 ev begrenzt

\subsection{Verschiede Arten}
Monokristalines Silizium 16-22\%
am besten, aber auch teuer
Polykristalines Silizium 15-16\%
verbreiteter, da billiger
Amporphes Silizium 	6-8\%

\subsection{Neue Wege}
\begin{itemize}
	\item Anschlüsse von hinten
	\item Oberflächentexturierung zur Flächen Maximierung z.B. auf-ätzen
	\item Bündeln mit z.B. Linsen
	\item Mehrere Zellen für verschiedene Wellenlängen hintereinander
\end{itemize}s
	
	% Kapitel 10
	\section{Kapitel 10 - Bipolartransistor }
	\todo{Wichtige Begriffe erklären}


\subsection{Bipolartransistor versus Feldeffekttransistor}
	\subsubsection{Einfluss der Skalierung}
	\subsubsection{Grundeigenschaften}
	\subsubsection{Anwendungsfelder}
\subsection{Bipolartransistor (Homojunktion, BJT)}
	\subsubsection{npn versus pnp}
	\subsubsection{Kenngrößen}
	\subsubsection{Funktionsprinzip}
	\subsubsection{Grundschaltungen}
\subsection{Heterojunktion Bipolartransistor (HBT)}
	\subsubsection{Aufbau}
	\subsubsection{Vorteile gegenüber BJT}
	\subsubsection{Erreichbare Leistungen}
	\subsubsection{Modulare Integration}


\todo{Fragen aus Own Clowd zuordnen}
\todo{Gruppenübungs-Inhalte ergänzen}
	
	% Kapitel 11
	\section{Kapitel 11 - Verbindungshalbleiter }
	\todo{Wichtige Begriffe erklären}
\subsection{Element- und Verbindungshalbleiter}

	Zunächst wird zwischen Element und Verbindungshalbleitern unterschieden. Bei den Elementhalbleitern handelt es sich um Silizium und Germanium. Dies sind beides Halbleiter der 4. Hauptgruppe (\ref{11_perioden}). Verbindungshalbleiter werden aus einer Verbindung von Elementen der 2. bis 6. Gauptgruppe zusammengesetzt. Einige Vertreter sind z.B. Gallium-Arsenid(GaAs) als Mischung von 3. und 5. Gruppe, Cadmium-Sulfid(CdS) als Verbindung von 2. und 6. und Silizium-Carbid(SiC) aus den Gruppen 4 und 6. Es gibt auch Verbindungshalbleiter die aus einer Mischung aus Germanium und Silizium bestehen. Dies kann für verspanntes Silizium verwendet werden. 

	\begin{figure}[h]
		\centering
		\includegraphics[width=0.5\textwidth]{Kapitel/Kap11/periodensystem.png}
		\caption{Ausschnitt der 2. bis 6. Hauptgruppe des Periodensystems der Elemente}
		\label{11_perioden}
	\end{figure}

	Die Struktur entspricht einem Diamantgitter. Sie tritt sowohl bei Element, als auch bei Verbindungshalbleitern auf. Bei letzteren wird sie allerdings auch Zinkblende-Struktur genannt.

	\subsubsection{Direkte und indirekte Halbleiter}
	Direkte und indirekte Halbleiter unterscheiden sich in der Position des Minimums / Maximums im Bänderdiagramm. Bei direkten Halbleitern ist ein Übergang vom Valenz in das Leitungsband direkt möglich. Bei indirekten hingegen muss neben der direkten Energiedifferenz auch eine Änderung des Impuls erfolgen (\ref{11_vergleich}). Dieses benötigt allerdings viel Energie, weshalb optische Bauelement, die z.B. leuchten sollen, nur aus direkten Halbleitern gebaut werden können. Die Energie, die für einen indirekten Übergang notwendig wären, liegen nicht mehr im für uns sichtbaren Licht.
	
	\begin{figure}[h]
		\centering
		\includegraphics[width=0.8\textwidth]{Kapitel/Kap11/direkte_indirekte_halbleiter.PNG}
		\caption{Direkter und indirekter Halbleiter im Vergleich}
		\label{11_vergleich}
	\end{figure}

	\begin{table}[h]
		\centering
		\begin{tabular}{c|c|c}
			&Direkt & Indirekt \\
			\hline
			EVB(max) und ELB(min) bei & gleichem k & verschiedenen k \\
			Impulsänderung beim Übergang & nein & notwendig \\
			Übergangs- und damit Rekombinationswahrscheinlichkeit & hoch & niedrig \\
			Lebensdauer angeregter Ladungsträger & kurz & lang\\
			Optoelektronik(LED, Laser usw.) & ja & nein\\
		\end{tabular}
		\caption{Unterschiede Direkter und indirekter Halbleiter}
		\label{11_unterschiede}
	\end{table}
	
	\subsubsection{Dotieren von Verbindungshalbleitern}	
	Das Dotieren von Elementhalbleitern ist relativ einfach. Bei Verbindungshalbleiter ist dies aber nicht so. Nehmen wir das Beispiel GaAs. Hier wird z.B. mit Silizium und Kohlenstoff dotiert. Ersetzt ein Si-Atom ein Ga-Atom, so komm es zu einer n-Dotierung. Ersetzt es hingegen ein As-Atom, so ist dies eine p-Dotierung. Dies ist jedoch technische schwer umzusetzen. Einfach ist es, wenn kleinere Atome nur größere ersetzten. Eine p-Dotierung wäre mit Kohlenstoff-Atomen möglich, ist aber oftmals schwieriger.
	
	\paragraph{Antiphasengrenze} Bei Verbindungshalbleitern tritt ein neuer Defekt auf. Möchte man z.B. GaAs auf eine Siliziumoberfläche aufbringen, kann es passieren, dass durch Unebenheiten des Siliziums Antiphasengrenzen im ausgetragenen Halbleiter entstehen (\ref{11_antiphase}).
	
	\begin{figure}[h]
		\centering
		\includegraphics[width=0.5\textwidth]{Kapitel/Kap11/antiphasen.PNG}
		\caption{Antiphasengrenzen im GaAs}
		\label{11_antiphase}
	\end{figure}
	
	\subsubsection{Ladungsträgerbeweglichkeiten}
	
	Die Ladungsträgerbeweglichkeiten unterscheiden sich in Unterschiedlichen Materialien stark (\ref{11_ladung}). So hat GaAs eine erheblich höhere Beweglichkeit von Elektronen, die von Löchern ist allerdings noch niedriger als die von Silizium. 
	\begin{table}[h]
		\centering
		\begin{tabular}{c|c|c}
			& $\mu_n $ & $\mu_p$ \\
			\hline
			Si & 1430 & 505 \\
			Ge & 3900 & 1900 \\
			GaAs & 8000 & 400 \\
			Cu & 32 & -\\
		\end{tabular}
		\caption{Ladungsträgerbeweglichkeit in unterschiedlichen Materialien}
		\label{11_ladung}
	\end{table}
	\paragraph{HEMT} \todo{Coming soon}
\subsection{Welleneigenschaften}
	\subsubsection{Monochromatisch}
	Monochromatisches Licht besteht nur aus einem einzigen spektralen Anteil. Somit das das Licht eine bestimmte Energie. Es ist sozusagen gleichfarbig.
	\subsubsection{kohärent} 
	Kohärenz bedeutet Phasengleichheit. Dies ist vor allem dann wichtig, wenn es zu Interferenz kommen soll. Ein guter Laser erzeugt kohärentes Licht. 
	\subsubsection{kollimiert}
	Bei kollimierten Licht verlaufen alle Lichtstrahlen parallel zueinander. Dies ist zum Beispiel bei einem Laser der Fall. Generelle geht man davon aus, dass auch unser Sonnenlicht kollimiert ist, da die Sonne eine derartig große Distanz zu uns aufweist, dass die Strahlen auf der Erde nahezu parallel ankommen.


\subsection{Leuchtdioden}
	\subsubsection{Funktionsprinzip}
	
	\paragraph{Aufbau} LEDs besitzen einen Napf, in dem einen Stück Halbleitermaterial eingebunden ist. Dieser Napf ist mit einem der beiden Pole verbunden und kann von außen angeschlossen werden. Der andere Pol wird über einen dünnen Bonddraht verbunden. Der gesamte Aufbau ist in Plastik eingegossen.
	
	\paragraph{Funktion} Bei dem Halbleitermaterial handelt es sich um einen pn-Übergang. Dabei ist die p-Schicht sehr löcherreich, dünn und auf der oberen Seite. Wird nun in Durchlassrichtung eine Spannung angelegt und ist diese groß genug, um die Bandlücke zu überwinden, so kommt es zu einer Überschwemmung der Grenzschicht mit freien Ladungsträgern. Bei der Rekombination (\ref{11_rekombination}) dieser wird Licht abgestrahlt. Dabei sollte es jedoch nur zur strahlenden Rekombination kommen. Auger, Störstellen und Oberflächenrekombination sollte weitestgehend vermieden werden. Die Helligkeit kann durch die Stromstärke geregelt werden. Wird sie zu heiß, kann sie ausfallen. Da die Farbe von der Bandlücke abhängt, gibt es für verschiedene Farben verschiedene Schwellspannungen.
	
	\begin{figure}
		\centering
		\includegraphics[width=0.5\textwidth]{Kapitel/Kap11/rekombination.png}
		\caption{Rekombination an LEDs}
		\label{11_rekombination}
	\end{figure}
	
	\subsubsection{Verschiedene Farben}
	
	Die Farbe wird durch die abgestrahlte Energie, und somit durch die Größe der Bandlücke bestimmt. Die Wellenlänge zu einer bestimmten Energie lässt sich mit folgender Formel berechnen:\\
	$\lambda(E_g) = \frac{hc}{E_g} = \frac{1240nm}{E_g}$\\
	Die Energie wird wie auch die Bandlücke in Elektronenvolt angegeben. Sie ist abhängig vom Halbleitermaterial (\ref{11_farben}).
	
	\begin{figure}
		\centering
		\includegraphics[width=0.5\textwidth]{Kapitel/Kap11/farben.png}
		\caption{Unterschiedliche Halbleiter und ihre Farben}
		\label{11_farben}
	\end{figure}
	
	\subsubsection{Weiße Dioden}
	
	Weißes licht kann auf mehrere Arten erstellt werden. Weißes licht ist eine Mischung aus mehreren Anteilen. So benötigt man für weißes Licht ein sehr Breitbandiges Lichtspektrum. Eine LED war jedoch gerade dafür bekannt relativ monochromatisch zu arbeiten. Zum einen kann eine Mischung aus RGB Anteilen stattfinden. Damit dies gut funktioniert müssen die einzelnen Farblichquelle sehr nahe beieinander liegen. Deshalb werden diese oftmals direkt in einem Bauteil zusammengefasst. Eine weitere Möglichkeit besteht daran, eine LED zu nehmen und das licht durch aufgelegte Stoffe zu beeinflussen. So kann eine UV-LED mit 3 unterschiedlichen Fluoreszenz-Farbstoffen in die RGB Anteile verschoben werden, die dann zusammen weißes licht ergeben. Eine billigere Variante Verwendet eine blaue LED, die mit nur einem Farbstoff ein gelb dazu mischt. Das hierbei entstehende Licht ist meistens sehr kalt, da der Blauanteil überwiegt.

	Generell lässt sich jedoch erkennen, dass sich die Farbspektren der unterschiedlichen Leuchtmittel stark unterscheiden (\ref{11_spektrum}). So hat eine Glühlampe einen sehr hohen Anteil an IR-Strahlung, was zu einem warmen (Wortwörtlich) Licht führt. Das Tageslicht enthält weniger rot, dafür mehr blau. Eine Leuchtstofflampe erzeugt nur einige genau definierte Spektralanteile. 
	
	\begin{figure}
		\centering
		\includegraphics[width=0.5\textwidth]{Kapitel/Kap11/spektrum.png}
		\caption{Das Licht der Sonne, einer Leuchtstofflampe, einer Glühlampe und einer LED im Vergleich}
		\label{11_spektrum}
	\end{figure}
	
\subsection{Laser}
\todo{machen}
	\subsubsection{Funktionsprinzip (stimulierte Emission, Pumpen, beteiligte Energieniveaus, ...)}
	
	\subsubsection{Resonatoren}
	\subsubsection{Halbleiterlaser (Prinzipeller Aufbau, VCSEL, ...)}


\todo{Fragen aus Own Clowd zuordnen}
\todo{Gruppenübungs-Inhalte ergänzen}
	
	% Kapitel 12
	\section{Kapitel 12 - Technologie }
	\todo{Wichtige Begriffe erklären}


\subsection{Technologieverfahren}
	\subsubsection{}
	\subsubsection{}
	\subsubsection{}
\subsection{}
	\subsubsection{}
	\subsubsection{}
	\subsubsection{}
\subsection{}
	\subsubsection{}
	\subsubsection{}
	\subsubsection{}
\subsection{}
	\subsubsection{}
	\subsubsection{}
	\subsubsection{}
\subsection{}
	\subsubsection{}
	\subsubsection{}
	\subsubsection{}

\todo{Fragen aus Own Clowd zuordnen}
\todo{Gruppenübungs-Inhalte ergänzen}
	
	% Kapitel 13
	\section{Kapitel 13 - Nanoelektronik }
	\subsection{Begrifflichkeit}
	Man unterscheidet die Begriffe Mirko- und Nanoelektronik. Sobald die Strukturen kleiner als 100nm groß sind befindet man sich im Bereich der Nanoelektronik. Jedoch kann man nicht immer einfach alles kleiner skalieren. aus diesem Grund müssen neue Technologien und Materialien verwendet werden.
\subsection{Showstopper für aquivaltes Skalieren}
	\begin{itemize}
		\item Physikalische Grenzen (z.B. Quanteneffekte)
		\item Grenzen der konventionellen Bauelementkonzepte
		\item Leistung und Wärme
		\item Grenzen der verwendeten Materialien
		\item Technische Gründe (z.B. Lithografie)
		\item Ökonomische Grenzen
		\item Grenzen beim Entwurf
	\end{itemize}
\subsection{FET}
	Für die Skalierung von Transistoren ist es notwendig, High-K Materialien als Isolator zu verwenden, damit der MOS Kondensator trotz dem geringen Größe immer noch einen ähnlichen Effekt aufweist. Außerdem gibt es ein Konzept, bei dem der Transistor nicht in die Tiefe aufgebaut ist, sondern auf der Fläche verteilt(KrisMOS). Man kann das Gate auch um einen Kanal herum aufbauen. Somit hat man eine große Einwirkung auf diesen.
\subsection{RC-Verzögerung}
	Nehmen wir an, wir haben 2 parallele Leiter. Soll sich ein Signal auf diesen verändern so kommt es durch kapazitive Effekte zu RC Verzögerungen. Jedoch sind viele Parameter durch das Design beschränkt, so lässt sich z.B. der Anstand zwischen den Leitungen nur schwer ändern. Aus diesem Grund ist es notwendig, Metallleitungen zu verwenden, welche einen sehr geringen Widerstand besitzen oder das Material zwischen den Leitern zu verändern.
\subsection{Zukünftige Elektronik}
	Es gitb mehrere Ideen, mit welchen die zukünftige Elektrik verbessert werden kann.
	\begin{itemize}
		\item Keine planaren, sonder dreidimensionale Chips
		\item Molekulare oder Kohlenstoffbasierte Elektronik
		\item Nanodräht
		\item Optische Kommunikation
	\end{itemize}

	
	
	% Beispiele 
	% Kapitel 0
	\section{Test}
	% in-line formeln
abdcdasd $\sum_{n=1}^{k}\rho_n$
% formeln als Block
\begin{align*} 
	\sum_{n=1}^{k}\rho_n
\end{align*}

%sehen unterschiedlich aus

\subsection{Thema 1}
	absasfasdf
\subsection{Thema 2}
	
	\begin{itemize}
		\item Halbleiter1
		\begin{enumerate}
			\item Montag
			\item Montag
			\item Montag
			\item Montag
		\end{enumerate}
		\item Halbleiter2
	\end{itemize}
	\includegraphics[width=\linewidth]{Kapitel/Kap01/test.png}
	\begin{center}
		
	
		\includegraphics[width=0.4\linewidth]{Kapitel/Kap01/test.png}
	\end{center}


\begin{tabular}{|l|r|}
	\hline
	\textbf{Hello} & World\\
	\hline
	Hello & Martin\\
	\hline
\end{tabular}
\todo{Hello WOlrd}

\subsection{Verteilungsfunktion}
\subsubsection{Fermiverteilung}


Definition der Fermi-Energie
Lage des Fermi-Niveaus (intrinsisch vs. dotiert)
Effektive Zustandsdichten
Ladungsträger im Halbleiter
Massenwirkungsgesetz
Neutralitätsbedingung
Intrinsische Ladungsträgerkonzentration
Bezeichnung von dotierten Halbleitern
Majoritäten und Minoritäten
Ladungsträgerbewegung
Driftstrom, Sättigung usw.
Diffusionsstrom
Temperaturspannung
Leitfähigkeit von Halbleitern
p- und n-Typ, Temperaturabhängigkeit usw.
Definitionen von Dotierniveaus


% geometry package --> Seitenränder einstellen
	
	
	
	
	
	% Aufgabe 2
	%\section{}
	%\input{./Kapitel/Aufgabe2}
	
	% Aufgabe 3
	%\section{}
	%\input{./Kapitel/Aufgabe3}
	
\end{document}