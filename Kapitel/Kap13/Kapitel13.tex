\subsection{Begrifflichkeit}
	Man unterscheidet die Begriffe Mirko- und Nanoelektronik. Sobald die Strukturen kleiner als 100nm groß sind befindet man sich im Bereich der Nanoelektronik. Jedoch kann man nicht immer einfach alles kleiner skalieren. aus diesem Grund müssen neue Technologien und Materialien verwendet werden.
\subsection{Showstopper für aquivalentes Skalieren}
	\begin{itemize}
		\item Physikalische Grenzen (z.B. Quanteneffekte)
		\item Grenzen der konventionellen Bauelementkonzepte
		\item Leistung und Wärme
		\item Grenzen der verwendeten Materialien
		\item Technische Gründe (z.B. Lithografie)
		\item Ökonomische Grenzen
		\item Grenzen beim Entwurf
	\end{itemize}
\subsection{FET}
	Für die Skalierung von Transistoren ist es notwendig, High-K Materialien als Isolator zu verwenden, damit der MOS Kondensator trotz dem geringen Größe immer noch einen ähnlichen Effekt aufweist. Außerdem gibt es ein Konzept, bei dem der Transistor nicht in die Tiefe aufgebaut ist, sondern auf der Fläche verteilt(KrisMOS). Man kann das Gate auch um einen Kanal herum aufbauen. Somit hat man eine große Einwirkung auf diesen.
\subsection{RC-Verzögerung}
	Nehmen wir an, wir haben 2 parallele Leiter. Soll sich ein Signal auf diesen verändern so kommt es durch kapazitive Effekte zu RC Verzögerungen. Jedoch sind viele Parameter durch das Design beschränkt, so lässt sich z.B. der Anstand zwischen den Leitungen nur schwer ändern. Aus diesem Grund ist es notwendig, Metallleitungen zu verwenden, welche einen sehr geringen Widerstand besitzen oder das Material zwischen den Leitern zu verändern.
\subsection{Zukünftige Elektronik}
	Es gibt mehrere Ideen, mit welchen die zukünftige Elektrik verbessert werden kann.
	\begin{itemize}
		\item Keine planaren, sonder dreidimensionale Chips
		\item Molekulare oder Kohlenstoffbasierte Elektronik
		\item Nanodräht
		\item Optische Kommunikation
	\end{itemize}
