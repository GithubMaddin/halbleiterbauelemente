

\todo{In schönere Form bringen}

\subsection{Wichtige Daten}
\begin{itemize}
	\item Edison Effekt (1883)\\
	Gleichspannungsregler mit Vakuumdiode
	\item Triode (1920er-1930er)\\
	Glühfaden emitiert Elektronen, die beschleunigt durch ein Gitter treten und durch dieses gesteuert werden können.\\
	Hohe Stromverstärkung\\
	Hoher Stromverbrauch
	\item Feldeffekt Transistor nach Lilienfeld (1928)\\
	Konnte noch nicht herstellen, aber die Funktionsweise wurde beschrieben.\\
	Ge-Punktkontakt Transistor (1947)\\
	Current Transfer + resistor = Transistor
	\item MOSFET (1960)
	\item Erster Mikroprozessor (1971)
	\item Erster PC (1981)
\end{itemize}

\subsection{Moores Gesetz (1965)}
Alle 18 Monate verdoppelt sich die Anzahl der Transistoren pro Chip
Ableitung der heute gültigen Form im Rythmus von 2 Jahren.

\subsection{Skalieren}
Nach Moores Gesetzt halbiert sich die Fläche
=> Dimensionierung ändert sich um Faktor $1 \sqrt{2}$
Technologiegenerationen: 180 130 90 65 45 32 22 ...
Alte Technologien bleiben trotzdem erhalten
Betriebsspannung skaliert mit, zunächst Faktor 0.7 dann 0.85
Geschwindigkeit steigt exponentiell an
Preis pro Transistor fällt exponentiell ab
Preis einer Chipfabrik verdoppelt sich alle 3 Jahre

\subsection{Roadmap}
Von ITRS aufgestellt (Hersteller, Zulieferer, Forschung)
Gliedert in machbare, mit Forschung machbare und nicht nicht machbare Optionen
Man sieht eine "rote Mauer", die schon immer existiert hat und die sich alle 5-10 Jahre durch Fortschritt nach hinten verschiebt.

\subsection{Technologie}
Alles unter 100nm ist Nanotechnologie
MOS Logic und MOS Mikroprozessoren überliegen im Umsatzanteil.
Mehr als 90% Auf Silizium und mehr als 80% intigriert

\subsection{Gliederung}

Elektronische Bauelemente
Passive Bauelemente
Spulen
Widerstände
Induktivitäten
Aktive Bauelemente
Analoge Bauelemente
Digitiale Bauelemente
Kontakt- und Verbindungselemente
Leiterplatten
Schalter
Steckverbinder
Optoelektronische Bauelemente
Sender
Leuchtdiode
Halbleiter-Laser
Empfänger
Detektroren
Solarzellen
