\subsection{Wichtige Daten}
\begin{itemize}
	\item Edison Effekt (1883)
	\begin{itemize}
		\item Gleichspannungsregler mit Vakuumdiode
	\end{itemize}
	\item Triode (1920er-1930er)
	\begin{itemize}
		\item Glühfaden emittiert Elektronen, die beschleunigt durch ein Gitter treten und durch dieses gesteuert werden können.
		\item Hohe Stromverstärkung
		\item Hoher Stromverbrauch
	\end{itemize}
	\item Feldeffekt Transistor nach Lilienfeld (1928)
	\begin{itemize}
		\item Konnte noch nicht herstellen, aber die Funktionsweise wurde beschrieben.
		\item Ge-Punktkontakt Transistor (1947)
		\item Current Transfer + Resistor = Transistor
	\end{itemize}
	\item MOSFET (1960)
	\item Erster Mikroprozessor (1971)
	\item Erster PC (1981)
\end{itemize}

\subsection{Moores Gesetz (1965)}
Alle 18 Monate verdoppelt sich die Anzahl der Transistoren pro Chip
Ableitung der heute gültigen Form im Rhythmus von 2 Jahren.

\subsection{Skalieren}
Dementsprechend halbiert sich die Fläche eines Transistors in diesem Zeitbereich. Darua slässt sich die Änderung der Dimensionierung (Längen) Berechnen: Sie verändert sich um den Faktor $1 \sqrt{2}$.
Daraus folgen auch die verwendeten Technologiegenerationen in nm: 180 130 90 65 45 32 22 ...
Obwohl es immer wieder neue kleine Technologien gibt, bleiben die Alten weiterhin auf dem Markt und werden produziert. Dies liegt daran, dass eine neue Technologie und die Herstellung dieser sehr teuer und auch nicht immer notwendig ist. Preis einer Chipfabrik verdoppelt sich alle 3 Jahre.
Neben der Größe skaliert zunächst auch die Betriebsspannung mit. Dabei verringerte diese sich am Anfang um die Faktor 0.7, zuletzt dann um den Faktor 0.85. Dies ist jedoch nicht unendlich lange möglich, da Bei bestimmten Spannungen z.B. die Schwellspannung eines Transistors zu hoch ist. 
Die Geschwindigkeit steigt exponentiell an und der Preis pro Transistor fällt exponentiell ab.

\subsection{Roadmap}
Die ITRS stellt eine Roadmap auf (Hersteller, Zulieferer, Forschung). Diese Gliedert Größen in machbare, mit Forschung machbare und nicht nicht machbare Optionen. Man sieht eine "rote Mauer", die schon immer existiert hat und die sich alle 5-10 Jahre durch Fortschritt nach hinten verschiebt.

\subsection{Technologie}
Alles unter 100nm ist Nanotechnologie. MOS Logic und MOS Mikroprozessoren überwiegen im Umsatzanteil.
Mehr als 90\% auf Silizium und mehr als 80\% intrigiert.

\subsection{Gliederung}

\subsubsection{Elektronische Bauelemente}
\begin{itemize}
	\item Passive Bauelemente
	\begin{itemize}
		\item Spulen
		\item Widerstände
		\item Induktivitäten
	\end{itemize}
	\item Aktive Bauelemente
	\begin{itemize}
		\item Analoge Bauelemente
		\item Digitiale Bauelemente
	\end{itemize}
	\item Kontakt- und Verbindungselemente
	\begin{itemize}
		\item Leiterplatten
		\item Schalter
		\item Steckverbinder
	\end{itemize}
\end{itemize}
\subsubsection{Optoelektronische Bauelemente}
\begin{itemize}
	\item Sender
	\begin{itemize}
		\item Leuchtdiode
		\item Halbleiter-Laser
	\end{itemize}
	\item Empfänger
	\begin{itemize}
		\item Detektroren
	\end{itemize}
	\item Solarzellen
\end{itemize}


