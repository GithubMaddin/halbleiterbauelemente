

\todo{Wichtige Begriffe erklären}


\subsection{Nutzung von Sonnenenergie}

	\subsubsection{Gegenwärtige Energiequellen}

	\subsubsection{Mögliche Entwicklungen}


\subsection{Funktionsweise einer Solarzelle}

	\subsubsection{Grundprinzip}

	\subsubsection{Warum braucht man eine pn-Struktur}

	\subsubsection{Wirkungsgrade, Ursachen für Verluste}

\subsection{Typen von Solarzellen}

	\subsubsection{Verschiedene Materialien}

	\subsubsection{Verschiedene Konstruktionen}




\subsection{Auslaufen der Energie Reserven}

Kohle ca. 100 Jahre

Erdgas und Erdöl ca. 50 Jahre

Uran ca. 70 Jahre


\subsection{Anteile an Erneuerbaren Energie}

Vor allem Skandinavien

Deutschland 38,5\% am gesamten Strommix


\subsection{Massendefekt}

Masse $\leftrightarrow$ Energie: $E = m\cdot c^2$

Massendefekt = Bindungsenergie

Diese Energie wird bei Kernumwandlungen freigesetzt


\subsection{Kernfusion}

$Deuterium H^3 + Tritium H^2 \rightarrow Heliumkern He^2 + n$


\subsection{Geschichte}

Erste Solarzelle (1954)

4-6\% Wirkungsgrad


\subsection{Vorteile Solarenergie}

Unbegrenzt vorhanden

Emissionslos

Reduzierung von energiepolitischen Abhängigkeiten


\subsection{Nachteile Solarenergie}

Nicht konstant (Wetter, Jahreszeiten, Tageszeiten...)

Herstellung nicht emissionsfrei

Hohe Kosten


\subsection{Brandenburg ist toll}


\subsection{Aufbau}

95% aler Solarzellen aus Silizium

np-Halbleiter Aufbau

n Schicht ist sehr dünn, damit das Licht an den pn-Übergang gelangen kann


\subsection{Photoelektrischer Effekt}

Äußerer (nicht für Solarzellen relevant)

Aufgeladene Oberfläche gibt Elektronen bei Bestrahlung ab

Innerer (relevant für Solarzellen)

besser Leitfähigkeit bei Beleuchtung, da Elektronen auf höheres Valenzband gehoben werden


\subsection{Ablauf}

Licht sorgt dafür, dass Elektronen in die n Schicht und Löcher in die p Schicht wandern. Können an Kontakten abgegriffen werden.

Elektronen-Loch-Paare müssen vor der Rekombination getrennt werden

Spannung immer ähnlich (0.5 V bei Silizium), Strom steigt mit Beleuchtungsstärke an

Leistung ist temperaturabhängig


$Stromabgabe = Generation - Rekombination = J = J_SC - J_0 \bigg(e^{\frac{qV}{kT}}-1\bigg)$

Rekombination = Dunkel-Diodenstrom

Keine Annahme über pn- Übergang für die Formale notwendig


Der pn-Übergang ist für die Kontaktierung. Die Fermi.Level der Metallkontakte richtet sich nach dem der angrenzenden Majoritäten

Somit könne die Löcher durch p und Elektronen durch n Halbleiter kontaktiert werden. Fließen nach außen ab.


Kennlinie wie bei einer Diode, aber verschiebt sich bei Sonneneinstrahlung


Maximum Power Point bei P=J*V = max

Füllfaktor ist Verhältnis zwischen Leistung und MPP


\subsection{Wirkungsgrad}

MPP / Lichtintensität

Photonen nicht energetisch genug ($hv < E_g$)

Photonen haben Überschussenergie die in Wärme umgesetzt wird ($hv > E_g$)

Abschattung durch Kontakte

Widerstandsverluste

Es kann nicht das gesamte Spektrum genutzt werden

Theoretisch auf 32,2\% bei Licht von 1,12 ev begrenzt


\subsection{Verschiede Arten}

Monokristalines Silizium 16-22\%

am besten, aber auch teuer

Polykristalines Silizium 15-16\%

verbreiteter, da billiger

Amporphes Silizium 	6-8\%


\subsection{Neue Wege}

\begin{itemize}

	\item Anschlüsse von hinten

	\item Oberflächentexturierung zur Flächen Maximierung z.B. auf-ätzen

	\item Bündeln mit z.B. Linsen

	\item Mehrere Zellen für verschiedene Wellenlängen hintereinander

\end{itemize}s



\todo{Fragen aus Own Clowd zuordnen}

\todo{Gruppenübungs-Inhalte ergänzen}


